%%%%%%%%%%%%%%%%%%%%%%%%%%%%%%%%%%%%%%%%%
% Twenty Seconds Resume/CV
% LaTeX Template
%
% Original author:
% Carmine Spagnuolo (cspagnuolo@unisa.it) with major modifications by 
% Fan (casafan.yang@gmail.com)
%
% License:
% The MIT License (see included LICENSE file)
%
%%%%%%%%%%%%%%%%%%%%%%%%%%%%%%%%%%%%%%%%%

%----------------------------------------------------------------------------------------
%	PACKAGES AND OTHER DOCUMENT CONFIGURATIONS
%----------------------------------------------------------------------------------------

\documentclass[letterpaper]{twentysecondcv} % a4paper for A4
%----------------------------------------------------------------------------------------
%	 PERSONAL INFORMATION
%----------------------------------------------------------------------------------------

% If you don't need one or more of the below, just remove the content leaving the command, e.g. \cvnumberphone{}



\cvname{WESH WESH} % name
\cvjobtitle{Full Stack Developpeur} % Job title/career

\cvnumberphone{06 ** ** ** **} % Phone number
\cvmail{casafan.yang@gmail.com} % Email address
\cvgit{https://github.com/CasaFan} %Github address
\cvlocation{Grenoble, France}

%----------------------------------------------------------------------------------------

\begin{document}
\makeprofile % Print the sidebar

%----------------------------------------------------------------------------------------
%	 EDUCATION
%----------------------------------------------------------------------------------------
\section{Education}

\begin{twenty} % Environment for a list with descriptions
	\twentyitem
    	{2016 - 2018}
        {Master MIAGE}
        {\href{https://www.univ-grenoble-alpes.fr//}{Université Grenoble Alpes}}
        {}
        {Grenoble, France}
	\twentyitem
    	{2015 - 2016}
        {Licence informatique mention MIAGE}
        {\href{https://www.univ-grenoble-alpes.fr//}{Université Grenoble Alpes}}
        {}
        {Grenoble, France}
    \twentyitem
    	{2013 - 2015}
        {Licence informatique}
        {\mbox{}\hfill\href{https://www.univ-perp.fr/}{Université Perpignan Via Domitia}}
        {}
        {Perpignan, France}
	%\twentyitem{<dates>}{<title>}{<organization>}{<location>}{<description>}
\end{twenty}

%----------------------------------------------------------------------------------------
%	 EXPERIENCE
%----------------------------------------------------------------------------------------

\section{Expériences Professionnelles}

\begin{twenty} % Environment for a list with descriptions
	\twentyitem
    	{Oct 2017 - \\Août 2018}
        {Full Stack Web Développeur - CDD}
        {\href{https://www.phpnet.org/}{PHPNET FRANCE}}
        {Grenoble, France}
        {
        {\begin{itemize}
        \item Conception et développement d'un nouveau service e-mail sur les domaines de clients.
        \item Développement d'une interface de la gestion des clés SSH pour se connecter à l'hébergement.
        \item Amélioration d'un outil d'administration de DNS.
        \item Maintenance du panel clients et d'outils administrations.
    	\end{itemize}}
        }
        
    \twentyitem
   		{Mai 2017 - \\ Août 2017}
        {Développeur PHP - Stage}
        {\href{https://www.phpnet.org/}{PHPNET FRANCE}}
        {Grenoble, France}
        {
        {\begin{itemize}
        \item Développement d'un outil de gestion d'astreinte et adapatation aux navigateurs mobiles.
        \item Mise en place d'un framework de tests unitaires(Atoum) et automatisation avec git hooks.
        \item Réalisation d'un outil d'anti-spam pour améliorer le service mail.
    \end{itemize}}
        }
    \twentyitem
   		{Mars 2016 - \\ Mai 2016}
        {Projet de pratique Android}
        {\href{https://www.univ-grenoble-alpes.fr//}{Université Grenoble Alpes}}
        {}
        {
        \begin{itemize}
        \item Réalisation d'une application mobile de la gestion de ticket du parking en utilisant une web service REST en php.
    \end{itemize}
    	}
     \twentyitem
   		{Oct 2016 - \\ Dec 2016}
        {Projet de pratique AngularJS}
        {\href{https://www.univ-grenoble-alpes.fr//}{Université Grenoble Alpes}}
        {}
        {
        \begin{itemize}
        \item Développement d'une interface de control des médias (dans un réseau local).
    \end{itemize}
    	}
        
    \twentyitem
   		{Mai 2016 - \\ Août 2016}
        {Concepteur, Développeur PHP - Stage}
        {\href{https://www.orha.fr/}{ORAH}}
        {Narbonne, France}
        {
        {\begin{itemize}
        \item Conception et développement d'un système de questionnaire paramétrable dans un équipe de deux personnes.
        \item Intégration de ce questionnaire dans une autre application web.
    \end{itemize}}
        }
	%\twentyitem{<dates>}{<title>}{<location>}{<description>}
\end{twenty}
\end{document} 
